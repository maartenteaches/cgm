\documentclass[a4,11pt]{article}
\usepackage{natbib}
\usepackage{a4wide}
\usepackage{setspace}
\linespread{1.1}
\author{Maarten L. Buis}
\title{Loophole or second chance? Track mobility and inequality of educational outcomes in Germany}
\begin{document}
\maketitle

An important distinction between educational systems is the distinction between sponsored and contest mobility. \citep{turner60} In the former pupils are selected into different tracks at an early age, while in the latter tracking is delayed for as long as possible. The German educational system is a good example of an educational system based on sponsored mobility: Traditionally, at the age of ten, children are assigned to one of three different types of secondary schools (Hauptschule, Realschule, Gymnasium). This early decision for a secondary school type has in turn important consequences for further educational options: Thus, a degree from a Realschule has gained increasing importance for securing a position in the German apprenticeship system; a degree from a Gymansium is (normally) the precondition for entering the tertiary educational system \citep{cortina_etal_08}. This institutional setting has been frequently criticized, not least because of the finding that this early educational decision and consequently life chances in many areas are not taken merely on a meritocratic basis, i.e. based on child’s endowments, but highly dependent on his or her parent’s socio-economic status \citep{breen_jonsson00,jacob_tieben09}. As a consequence, children from a disadvantaged background are more likely to end up in a school type that is below their ability \citep{Uhlig_etal09}.

The critique that the age of ten is too early to take such a long-term decision has led at the end of the 1960s to reforms of the West-German educational system improving the possibilities of moving between tracks, so called track mobility \citep{cortina_etal_08,jacob_tieben09}. For example, pupils can change from \emph{Realschule} (middle general secondary education) to \emph{Gymnasium} (higher general secondary education) or \emph{vice versa} before finishing their initial school type. Alternatively, someone can for example first finish her or his diploma in \emph{Realschule} and continue one year of general secondary education to also attain an \emph{Abitur}. Another example of downward track mobility is getting non-tertiary vocational training after attaining an \emph{Abitur}. 

The idea behind track mobility is that the early decision for a secondary school type does not necessarily have to be permanent; instead there should be possibilities at later stages to correct early mistakes \citep{Dustmann17}. However, there is empirical evidence that track mobility does not solve the problem that children from disadvantaged background tend to end up in lower school types: it is the privileged students that are disproportionally upwardly mobile and the less privileged students that are disproportionally downwardly mobile \citep{hillmert_jacob10,jacob_tieben09,maaz_etal04,neugebauer_schindler12,trautwein_etal08}. These studies have primarily looked at the consequences of these reforms for students of lower and higher socio-economic background. However, social status, parental education, or class are not the only forms of privilege. For example, there is also gender and migration background. 

Theoretically, we have no reason to believe that these different types of disadvantaged groups will respond in the same way to the opportunity of track mobility. High status children in lower tracks are motivated to change their track in order to avoid downward mobility compared to their parents. In early cohorts being in a lower track would have less consequences for women compared to men, as women were not expected to become the main breadwinner. So in those early cohorts women in lower tracks would face less pressure to change track than men. In more recent cohorts this has likely changed. Children with a migration background may face discrimination and the family may have less knowledge about the German education system, but at the same time they tend to have significantly higher educational aspirations than their German counterparts.

so the main question is: How does track mobility contribute to the advantage of children from better educated parents, children without migration background, and men in Germany in the second half of the twentieth century? 

I will try to answer this with the adult cohort from the National Educational Panel Study (NEPS) \citep{NEPS_adult}. The key technical question is what `contribute to the advantage of privileged groups' means. The advantage is operationalized as the proportion of privileged people attaining a tertiary degree minus that proportion for not privileged people. In both groups some will have attained a tertiary degree directly and others through track mobility. The contribution of track mobility is the defined as how much the difference between privileged and non-privileged changes from those who attained the tertiary degree directly and everybody who attained a tertiary degree. 

\bibliography{tilburg}
\bibliographystyle{apalike}
\end{document} 